\subsection{Sprint 5}

\subsubsection{Introduction}
In this sprint, our focus was on completing the communication between the console and the game using JNI. Backend tasks were assigned for realizing user stories related to establishing communication between the game and the console, while frontend tasks included various implementations for the Pacman game.

\subsubsection{Spikes}

No spikes were conducted in this sprint.

\subsubsection{POCs}

No proofs of concept (POCs) were conducted in this sprint.

\subsubsection{Technical Justification}

During Sprint 0, the team exclusively focused on conducting spikes instead of developing POCs. This decision was based on the following reasons:

\subsubsection{High-Level Architecture (HLArchitecture)}

The architecture followed in this sprint was based on JNI communication between game, bridge, console, and screen.

\begin{itemize}
    \item Architecture: \href{https://github.com/Pending-Name-21/arquitecture/pull/12}{Link to architecture}
    \item C4 DSL Link: \href{https://structurizr.com/dsl}{Visualizer}
\end{itemize}

\newpage

\subsubsection{Product Backlog}

\paragraph{Done}
Completed spikes.

\begin{itemize}
    \item UML Pacman Game: \href{https://tree.taiga.io/project/joseluis-teran-coffeetime/us/72?milestone=395911}{US id-72}
    \item Execution Script: \href{https://tree.taiga.io/project/joseluis-teran-coffeetime/us/4?milestone=390348}{US id-176}
    \item Pacman Character: \href{https://tree.taiga.io/project/joseluis-teran-coffeetime/us/3?milestone=390348}{US id-79}
    \item Pacman Game Score: \href{https://tree.taiga.io/project/joseluis-teran-coffeetime/us/5?milestone=390348}{US id-84}
    \item Pacman Map: \href{https://tree.taiga.io/project/joseluis-teran-coffeetime/us/6?milestone=390348}{US id-82}
\end{itemize}

\paragraph{Carry Overs}
\begin{itemize}
    \item Tic Tac Toe Game: \href{https://tree.taiga.io/project/joseluis-teran-coffeetime/us/25?milestone=395911}{US id-25}
    \item Bridge Screen Communication: \href{https://tree.taiga.io/project/joseluis-teran-coffeetime/us/6?milestone=390348}{US id-135}
    \item Collisions on Pacman: \href{https://tree.taiga.io/project/joseluis-teran-coffeetime/us/3?milestone=390348}{US id-124}
    \item Pacman Game Initialization: \href{https://tree.taiga.io/project/joseluis-teran-coffeetime/us/110?milestone=395911}{US id-110}
    \item Game Loop Conditions: \href{https://tree.taiga.io/project/joseluis-teran-coffeetime/us/5?milestone=390348}{US id-166}
    \item Pacman Collectables: \href{https://tree.taiga.io/project/joseluis-teran-coffeetime/us/5?milestone=390348}{US id-166}
\end{itemize}

\subsection{Impediments}
For the impediments we develop a word file:

\href{https://docs.google.com/spreadsheets/d/1S3ndUFktff6ETyNhOyirIFNed71W4ApTLGyjX8xSzUQ/edit?usp=sharing}{impediments}

\subsection{Conclusions}

In Sprint 5, our focus was on achieving communication between the game and the console, but the goal was not achieved in this sprint. The most significant progress was seen in the realm of games, specifically in the development of the Pacman Game.

\subsubsection{action items}

\begin{itemize}
    \item Add priority as tags in the US to make it more visible
\end{itemize}


\subsubsection{Epics}

For Sprint 5, we have the following epics:

\begin{itemize}
    \item Screen
    \item Bridge
    \item Console
    \item Pacman Game
    \item Tic Tac Toe Game
\end{itemize}