\subsection{Architecture changes}

\subsubsection{Introduction}
software project plays a pivotal role in determining its success and sustainability. A well-defined architecture not only lays the foundation for the development process but also ensures that the system is scalable, maintainable, and adaptable to future requirements.
\subsubsection{Architecture Evolution}
We will be reviewing the different changes that the architecture has had throughout the development of the project, which were two in total, we will delve into more details below.
\paragraph{Basic Architecture}
First version of our architecture, these are the first concepts we had to be able to synthesize the idea of communication between console and video game based on an approach with LLVM given the capabilities of the tool such as having a run-time, the ability to go from a high-level language to a low level.

\begin{itemize}
    \item First Architecture concept: \href{https://github.com/Pending-Name-21/arquitecture/pull/1/files}{Architecture}
\end{itemize}

\paragraph{Architecture bassed on JNI}
Based on the spikes made in sprint 0 regarding LLVM and a JNI alternative, it was seen that although it was possible to use LLVM for our purposes, the tool was not made for communication between languages, LLVM is a set of tools focused on creating of compilers, although it had a tool developed for what we wanted, it was deprecated and abandoned, so LLVM was declared unviable, JNI as the alternative and based on the results of the JNI spike that was successful and showed a continuous communication between Java and Rust, it was decided to opt for an approach with JNI.

\begin{itemize}
    \item Second Architecture concept: \href{https://github.com/Pending-Name-21/arquitecture/pull/5}{Architecture}
    \item Spike JVM: \href{https://tree.taiga.io/project/joseluis-teran-coffeetime/us/2?milestone=390348}{JNI}
\end{itemize}

\subsubsection{Actual Architecture}
It was observed that the communication that would be formed between console and bridge in terms of GUI was not clear, so for the game to run properly, the GUI should be running along with the game; and the current approach ties the graphical execution to the console, where each routine would have to create a GUI instance in order to fulfill its needs.

\begin{itemize}
    \item Last Architecture concept: \href{https://github.com/Pending-Name-21/arquitecture/pull/12}{Architecture}
\end{itemize}
