\section{Practical Framework}
% \author{}
% \date{}

% \maketitle

\subsection{Sprint 1}

\subsubsection{Introduction}

In this sprint, our focus was on starting the contract for the developer and 
keep investigation about execution for the game using the console we define.
That's why we started with UMLs for our first game (Tic Tac Toe), console and
our contract (frontend-library), that is the Bridge.
At the same, we had been working on some Spikes for the console, such as to
define a graphic library for it, communication via JNI, apply shared memory 
between a game and console and finally, register input events, like mouse and keyboard.
All these Skipes were related in the context of the technologies used in the project.

\subsubsection{UMLs}

We had been starting on the UMLs for the contract, Tic Tac Toe game and console.

\begin{itemize}
    \item Task ID - 17: \href{https://tree.taiga.io/project/joseluis-teran-coffeetime/us/17?milestone=392128}{Console Diagram on Taiga (Comments Section)}
    \item Task ID - 18: \href{https://github.com/Pending-Name-21/arquitecture/pull/2}{Library UML Diagram on GitHub}
    \item Task ID - 22: \href{https://github.com/Pending-Name-21/arquitecture/pull/3}{Tic Tac Toe UML Diagram on GitHub}
\end{itemize}

\subsubsection{Spikes}

As we mentioned before, there were some Spikes to work with and here is the documentation generated by them:

\begin{itemize}    
    \item Task ID - 19: \href{https://docs.google.com/document/d/1a6wyQA0LM5thyAfOsWnkG-fTLcThvrsVPOGmFcoqkvw/edit?usp=sharing}{Graphic Libraries Documentation on Google Docs}
    \item Task ID - 20: \href{https://github.com/Pending-Name-21/console/tree/vm-spikes/jni_spike}{JNI Documentation on GitHub}
    \item Task ID - 21: \href{https://github.com/Pending-Name-21/console/blob/vm-spikes/shared-memory/README.md}{Shared Memory Documentation on GitHub}
    \item Task ID - 24: \href{https://universidadsalesian-my.sharepoint.com/:w:/g/personal/axel_ayala_9412013_usalesiana_edu_bo/EZlHobuXqW5AmffmDNnGaKYBdpordz1QlVJk88Pe_6S7HQ?e=DymfMq}{Input Listener Documentation on Word}
\end{itemize}

\subsubsection{POCs}

In this Sprint, we had no Proofs of Concept (POCs).

\subsubsection{Technical Justification}

During Sprint 1, the team exclusively focused on designing UMLs and
generating Spikes for console. This decision was based on the following reasons:

\subsubsection{High - Level Architecture}

The architecture followed in this sprint was based on the initial architecture.

\begin{itemize}
    \item Architecture: \href{https://github.com/Pending-Name-21/arquitecture/pull/1/files}{Architecture Initial Version}
    You can see it on: 
    \item C4 DSL: \href{https://structurizr.com/dsl}{Visualizer}
\end{itemize}

\newpage

\subsubsection{Epics}

For this Sprint 1, we have the following epics:

\begin{itemize}
    \item Bridge
    \item Console
    \item Graphics
    \item Games
\end{itemize}

\subsubsection{Sprint Backlog}

\paragraph{Done}

Completed Spikes and User Stories.
\begin{itemize}
    \item Task ID - 19: \href{https://tree.taiga.io/project/joseluis-teran-coffeetime/us/19?milestone=392128}{Spike Graphics Libraries for Console}
    \item Task ID - 20: \href{https://tree.taiga.io/project/joseluis-teran-coffeetime/us/20?milestone=392128}{Spike Console with JNI}
    \item Task ID - 21: \href{https://tree.taiga.io/project/joseluis-teran-coffeetime/us/21?milestone=392128}{Spike Shared Memory between Game - Console}
    \item Task ID - 24: \href{https://tree.taiga.io/project/joseluis-teran-coffeetime/us/24?milestone=392128}{Spike Mouse and Keyboard Events Listener}
    \item Task ID - 18: \href{https://tree.taiga.io/project/joseluis-teran-coffeetime/us/18?milestone=392128}{UML Library}
    \item Task ID - 22: \href{https://tree.taiga.io/project/joseluis-teran-coffeetime/us/22?milestone=392128}{UML Tic Tac Toe Game}
    \item Task ID - 17: \href{https://tree.taiga.io/project/joseluis-teran-coffeetime/us/17?milestone=392128}{UML Console}
    \item Task ID - 30: \href{https://tree.taiga.io/project/joseluis-teran-coffeetime/us/30?milestone=392128}{Frontend Library Implementation}
    \item Task ID - 35: \href{https://tree.taiga.io/project/joseluis-teran-coffeetime/us/35?milestone=392128}{Tic Tac Toe Mockups and Sprites}    
\end{itemize}

\paragraph{Conclusions}

In Sprint 1, our focus was give that initial step in the game and the console, 
also of reviewing some documentation for our spikes. The most important goals
achieved were complete the UML for the contract and console too, besides of 
defining the graphic library, working with JNI, register input events and apply 
shared memory between the game and console.
